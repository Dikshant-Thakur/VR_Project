\section{State of Art}
The combination of virtual reality (VR) simulations and autonomous systems is revolutionizing industrial safety by enabling more efficient hazard detection and response. These technologies are particularly crucial in high-risk environments like oil refineries, where traditional methods often fail to address the complexity and urgency of potential dangers.

\subsection*{Context and Importance}

Oil refineries face persistent risks, such as oil spills and structural failures, which require rapid and effective responses. Conventional safety systems are highly dependent on human intervention, leading to delays and inaccuracies under challenging conditions such as low visibility or night-time operations. The introduction of UAVs (Unmanned Aerial Vehicles), UGVs (Unmanned Ground Vehicles), and VR simulations offers innovative real-time solutions while reducing human exposure to dangerous tasks.

\subsection*{Advances in Research}

Recent developments highlight the transformative role of autonomous technologies in industrial safety. Thomas De Kerf and his team created a UAV system that uses infrared and RGB cameras with machine learning to detect oil spills, achieving 89 percent accuracy across diverse environments. This shows its potential for fast and reliable hazard identification.

Furthermore, Fabio Augusto de Alcantara Andrade and colleagues leveraged virtual reality environments to develop a high-fidelity simulation platform for testing autonomous systems in solar plant inspections. Built with Unreal Engine, this approach reduced the risks and costs associated with physical testing while providing a robust framework for refining autonomous systems. Applying such methods to refinery settings could optimize UAV-UGV coordination for hazard management.

\subsection*{Challenges and Opportunities}

Despite their immense potential, autonomous systems are not without challenges. Issues like sensor errors and interference from environmental factors can compromise their reliability. To address these limitations, integrating human oversight into autonomous operations remains essential. As the technology continues to evolve, its ability to function effectively in diverse and unpredictable conditions will ultimately determine its success in industrial applications.