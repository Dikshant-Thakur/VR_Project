\documentclass{article}

\usepackage{dibrisunige-report}
\usepackage{codespace}

% Sub-preambles
% https://github.com/MartinScharrer/standalone

% Encodings
\usepackage{amsmath,amssymb,gensymb,textcomp}

% Better tables
% Wide tables go to https://tex.stackexchange.com/q/332902
\usepackage{array,multicol,multirow,siunitx,tabularx}

% Better enum
\usepackage{enumitem}

% Graphics
\usepackage{caption,float}

% Allow setting >max< width of figure
% 'export' allows adjustbox keys in \includegraphics
\usepackage[export]{adjustbox}

% For demonstration purposes, remove in production
\usepackage{mwe}

% Configurations
\newcounter{memberrowno}
\setcounter{memberrowno}{0}

\ocoursename{VR FOR ROBOTICS}
\oreporttype{Project Report}
\title{Project title}
\oadvisor{Subhransu Sourav Priyadarshan}
\reportlayout%

% Custom commands
\newcommand*\mean[1]{\bar{#1}}

% Title and Metadata
\title{Dynamic Hazard Simulation and Autonomous Response in Industrial Environments}
\author{[Your Name(s)]}
\date{\today}

\begin{document}

% Cover Page
\coverpage%

\section*{Member list \& Workload}
\begin{center}
  \begin{tabular}{>{\stepcounter{memberrowno}\thememberrowno}llcc}
    
    \multicolumn{1}{c}{\textbf{No.}} & \textbf{Full name} & \textbf{Student ID} & \textbf{Percentage of work} \\
    
                                     & Josue Tinoco& 5835667& 25\%\\
                                     & Nima Abaeian& 5967579& 25\%\\
    
 & Dikshant Thakur&5943225 &25\%\\
 & Girum Molla Desalegn&6020433 &25\%\\
  \end{tabular}
\end{center}

\newpage
% Table of Contents
\tableofcontents
\newpage


% Abstract
%\section{Abstract (max 500 chars)}
%This is how you normally work with \LaTeX, but you can also split a project into smaller files for easier management.
%\\
%Here a few lines to describe the goal of the project, the  solution/s adopted and (in case) the difficulties faced and novelties proposed. 
%\\
%To import other files, you can use \mintinline{latex}{\input{}} or \mintinline{latex}{\include{}}.
%There differences can be found at \url{https://tex.stackexchange.com/a/250}, but in short
%
%\[\mintinline{latex}{\include{filename}} = \mintinline{latex}{\clearpage \input{filename} \clearpage}\]

% Concept Overview
\section{Concept Overview}
\begin{itemize} 
    \item \textbf{Brief Description:} "Dynamic Hazard Simulation and Autonomous Response in Industrial Environments" is a project focused on designing a realistic simulation of an industrial refinery environment. It includes dynamic hazard detection and immediate response systems using both UAVs (Unmanned Aerial Vehicles) and UGVs (Unmanned Ground Vehicles), that can work in perfect synchronization, with custom developed real-time communication. The project is designed to monitor floating roof storage tanks and respond autonomously to any detected hazards or anomalies, ensuring safety and efficiency in the industrial area.

    \item \textbf{Objectives:} 
    \\
1. Create a realistic simulation environment of an industrial refinery area using Unreal Engine.
\\
2. Code and simulate the movement of the floating roof storage tanks during operation, including the necessary fluid physics for the oil inflow and outflow.
\\
3. Develop a smart UAV and UGV system capable of detecting hazards in the storage tanks, and autonomously adjust their trajectory and response.
\\
4. Enable real-time communication and coordination between UAVs and UGVs for more efficient strategies.

    \item \textbf{Key Feature:} The integration of real-time hazard detection and autonomous response, managed by UAV and UGV collaboration. This is further enhanced by the dynamic simulation of floating roof storage tanks, which adds an additional layer of realism to the system.
\end{itemize}

% Target Audience
\section{Target Audience}
\begin{itemize}
    \item \textbf{Primary Users:} Industry engineers and safety managers, particularly those involved in refineries and other similar hazardous environments. 
\\
The project targets those relying on traditional, labor-intensive hazard detection methods, such as manual inspections or manually controlled UAVs. By automating hazard detection and response, we expect to demonstrate the efficiency and safety benefits of modern autonomous technologies tailored for their industry.
    \item \textbf{Secondary Users:} Students, researchers, and professional developers, interested in areas like industrial automation, robotics, or environmental safety.
\end{itemize}

% Example of Use Case
\section{Example of Use Case}
% monitoring_system.tex
\textbf{\Large Actors}
\begin{itemize}
    \item \textbf{Primary Actor:} Refinery Operator
    \item \textbf{Secondary Actors:} UAV, UGV
\end{itemize}

\textbf{\Large Basic Flow}
\begin{itemize}
    \item \textbf{Operator Initiates Monitoring:}
    \begin{itemize}
        \item Operator starts the monitoring system, which triggers the UAV to begin its patrol.
    \end{itemize}
%second point
    \item \textbf{UAV Patrols Refinery:}
    \begin{itemize}
        \item UAV autonomously flies predefined routes, capturing real-time video and sensor data.
        \item UAV uses computer vision algorithms to detect anomalies, such as:
        \begin{itemize}
            \item Unusual roof movement of floating storage tanks
            \item Spills or leaks
            \item Unauthorized personnel or vehicles
        \end{itemize}
    \end{itemize}
%third point
    \item \textbf{Anomaly Detection:}
    \begin{itemize}
        \item If an anomaly is detected, the UAV alerts the operator and the UGV.
    \end{itemize}
%fourth point
    \item \textbf{Anomaly Detection:}
    \begin{itemize}
        \item UAV adjusts its flight path to focus on the anomaly area, capturing detailed images and videos.
        \item UGV is dispatched to the ground-level location indicated by the UAV, using the real-time data to navigate.
    \end{itemize}
%fifth point
    \item \textbf{Operator Assessment and Response:}
    \begin{itemize}
        \item Operator reviews the data from the UAV and UGV to assess the severity of the situation.
        \item Based on the assessment, the operator may:
        \begin{itemize}
            \item Dispatch emergency response teams
            \item Activate safety protocols
            \item Implement corrective actions
        \end{itemize}
    \end{itemize}
%sixth point
    \item \textbf{Continuous Monitoring:}
    \begin{itemize}
        \item The UAV and UGV continue to monitor the situation, providing real-time updates to the operator.
        \item The system adapts to changing conditions, adjusting flight paths and ground-level inspections as needed.
    \end{itemize}
\end{itemize}



% Technical Information
\section{Technical Information}
\begin{itemize}
    \item \textbf{Technologies \& Software:} 
    \begin{itemize}
        \item Unreal Engine 4.2 for simulation environment development
        \item AirSim for UAV and UGV dynamics
        \item Python for programming core algorithms 
    \end{itemize}
    \item \textbf{Platforms:} 
    \begin{itemize}
        \item Desktop-based simulation for Windows
    \end{itemize}
    \item \textbf{Programming Languages:}
    \begin{itemize}
        \item Python and Blueprint scripting for Unreal Engine functionalities.
    \end{itemize}
    \item \textbf{Frameworks:}
    \begin{itemize}
        \item ROS (Robot Operating System) for robotics communication and coordination
        \item AirSim for UAV-UGV simulation and testing
    \end{itemize}

\end{itemize}

% Development Plan
\section{Development Plan}
\begin{itemize}
    \item \textbf{Team and Roles:}
    \begin{itemize}
        \item \textbf{Josue Tinoco: Refinery Environment and Floating Roof Tank Dynamics}  
        \begin{itemize}
            \item Builds the Unreal Engine refinery environment.
            \item Develops and codes the simulation model for the floating roof storage tanks.
            \item Focuses on the movement of the tank roof during oil inflow and outflow.
            \item Ensures high accuracy and realism in the tank geometry and behavioral simulation.
        \end{itemize}
        
        \item \textbf{Girum Molla Desalegn: UAV Model Development}
        \begin{itemize}
            \item Designs and implements the UAV model, including movement dynamics and sensor integration.
            \item Develops flight patterns for spill detection and tank monitoring.
            \item Ensures the UAV operates effectively within the existing VR refinery environment.
            \item Focuses on real-time anomaly detection and accurate monitoring capabilities.
        \end{itemize}
        
        \item \textbf{Nima Abaeian: UGV Model Development}
        \begin{itemize}
            \item Builds and programs the UGV model for ground-level navigation and response.
            \item Implements tracking mechanisms to respond to data received from the UAV.
            \item Ensures seamless operation and interaction with the UAV in the VR refinery.
            \item Includes adaptive ground-level inspection and hazard management strategies.
        \end{itemize}
        
        \item \textbf{Dikshant Thakur: Real-time UAV-UGV Communication and Integration}
        \begin{itemize}
            \item Develops and codes the real-time communication system between UAVs and UGVs.
            \item Enables UAVs to detect spills and tank movements effectively.
            \item Coordinates UGV responses based on real-time data shared by UAVs.
            \item Integrates both UAV and UGV models into the VR refinery environment to ensure smooth interactions and system functionality.
        \end{itemize}
    \end{itemize}
    \item \textbf{Timeline:} 
        \begin{itemize}
        \item \textbf{Concept}: Delivered on December 20th.
        \item \textbf{Prototype}: The first simulation prototypes of each section will be completed before the end of 2024.
        \item \textbf{Beta}: First half of January 2025. This will be a test heavy phase where any bugs or unintended actions can be corrected.
        \item \textbf{Launch}: Last week of January. The project will be ready for assessment.
        \end{itemize}
\end{itemize}


% State or Art
\section{State of Art}
The combination of virtual reality (VR) simulations and autonomous systems is revolutionizing industrial safety by enabling more efficient hazard detection and response. These technologies are particularly crucial in high-risk environments like oil refineries, where traditional methods often fail to address the complexity and urgency of potential dangers.

\subsection*{Context and Importance}

Oil refineries face persistent risks, such as oil spills and structural failures, which require rapid and effective responses. Conventional safety systems are highly dependent on human intervention, leading to delays and inaccuracies under challenging conditions such as low visibility or night-time operations. The introduction of UAVs (Unmanned Aerial Vehicles), UGVs (Unmanned Ground Vehicles), and VR simulations offers innovative real-time solutions while reducing human exposure to dangerous tasks.

\subsection*{Advances in Research}

Recent developments highlight the transformative role of autonomous technologies in industrial safety. Thomas De Kerf and his team created a UAV system that uses infrared and RGB cameras with machine learning to detect oil spills, achieving 89 percent accuracy across diverse environments. This shows its potential for fast and reliable hazard identification.

Furthermore, Fabio Augusto de Alcantara Andrade and colleagues leveraged virtual reality environments to develop a high-fidelity simulation platform for testing autonomous systems in solar plant inspections. Built with Unreal Engine, this approach reduced the risks and costs associated with physical testing while providing a robust framework for refining autonomous systems. Applying such methods to refinery settings could optimize UAV-UGV coordination for hazard management.

\subsection*{Challenges and Opportunities}

Despite their immense potential, autonomous systems are not without challenges. Issues like sensor errors and interference from environmental factors can compromise their reliability. To address these limitations, integrating human oversight into autonomous operations remains essential. As the technology continues to evolve, its ability to function effectively in diverse and unpredictable conditions will ultimately determine its success in industrial applications.

% other chapters for Final Report
%\section{Tools}

Here the enumerated list of software used, with version numbers and a readme file.\\
Please include reference to a GitHub Project here.\\
%\section{Description}

Here the description of the work done. Max 2 pages.
%\section{Results}

Here the main achieved results. Use in case tables, figures, etc.\\
Max 1 page
%\section{Conclusions}

Here recall the overall project done, and in case of findings some suggestions for future works.\\

Max 1000 chars.

% some example of chapters
%\section{Better tables}
The recommended way is by using the booktabs package and drop all vertical rules.

Tabularx is simply tabular but with X environment, meaning that it will try to use all of \mintinline{latex}{\linewidth}.

\begin{center}
  \begin{tabularx}{\textwidth}{l*{2}{X}}
    \toprule
         & OOP & FP \\
    \cmidrule(lr){2-3}
    Pros &     &    \\
         &     &    \\
         &     &    \\
    \midrule
    Cons &     &    \\
         &     &    \\
         &     &    \\
    \bottomrule
  \end{tabularx}
\end{center}

More information can be found at \url{https://latex-tutorial.com/tables-in-latex/}.

%\section{Better enumerator}
Normal enumerator gets the job done, but what if you want custom numbering?
This implementation allows custom labeling, either by pre-defined rules or in-place.

\begin{enumerate}[label={\alph*.yeah}]
  \item First item
  \item Second item
  \item[custom] Third item
\end{enumerate}

%\section{Codeblocks}
There are several ways to embed code in a \LaTeX{} file.
Here are inline code, embedded codeblock, and external import.

\begin{itemize}
  \item External import

  \inputcode[highlightlines={1,10-13}]{Python}{code/example.py}

  \item With custom line range

  \inputcode[firstline=10,lastline=13]{Python}{code/example.py}

  \item Embedded

  \begin{code}{Python}
  class iostream:
      def __lshift__(self, other):
          print(other, end='')
          return self

      def __repr__(self):
          return ''
  \end{code}

  \item Inline

  \mintinline{Python}{print('Hello, world!')}
\end{itemize}

You can also define your custom inline as \url{https://tex.stackexchange.com/a/148479}.

This is one way to input algorithms.

\begin{algorithm}[H]
  \caption{QL algorithm}
  Initialize \(Q\)-table values \((Q(s, a))\) arbitrarily\;
  Initialize a state \((s_t)\)\;
  Repeat Steps~\ref{alg:step_4} to~\ref{alg:step_6} until learning period ends\;
  Choose an action \((a_t)\) for the current state \((s_t)\) using an exploratory policy\; \nllabel{alg:step_4}
  Take action \((a_t)\) and observe the new state \((s_t + 1)\) and reward \((r_t + 1)\)\;
  Update \(Q\)-value\; \nllabel{alg:step_6}
\end{algorithm}

%\section{Figures with flexible width}

\hrule % to see \linewidth

\includegraphics[max width=0.9\linewidth]{example-image-1x1}

\includegraphics[scale=0.7]{example-image-a4}

With \mintinline{latex}{\adjincludegraphics} (or \mintinline{latex}{\adjustimage}) you can also use the original width as \mintinline{latex}{\width}:

\adjincludegraphics[width=\ifdim \width > \linewidth \linewidth \else \width \fi]{example-image}



\bibliographystyle{plain}
\bibliography{refs}
\nocite{*}
\end{document}
